\documentclass[numberedappendix]{aastex6}

\usepackage{graphicx}

\usepackage{hyperref}

\newcommand{\maestro}{{\sffamily Maestro}}
\newcommand{\castro}{{\sffamily Castro}}
\newcommand{\starkiller}{{\sffamily StarKiller}}
\newcommand{\starlord}{{\sffamily StarLord}}
\newcommand{\nyx}{{\sffamily Nyx}}
\newcommand{\amrex}{{\sffamily AMReX}}


\usepackage{color}
\setlength{\marginparwidth}{0.75in}
\newcommand{\MarginPar}[1]{\marginpar{\vskip-\baselineskip\raggedright\tiny\sffamily\hrule\smallskip{\color{red}#1}\par\smallskip\hrule}}



\begin{document}

\title{Preparing an AMR Library for Summit}
\shortauthors{Katz et al.}

\author{Max Katz\altaffilmark{1,2}}
\author{Maria Barrios Sazo\altaffilmark{2}}
\author{Brian Friesen\altaffilmark{3}}
\author{Adam Jacobs\altaffilmark{4}}
\author{Xinsheng Qin\altaffilmark{5}}
\author{Donald Willcox\altaffilmark{2}}
\author{Michael Zingale\altaffilmark{2}}

\altaffiltext{1}
{
  NVIDIA Corporation
}

\altaffiltext{2}
{
  Department of Physics and Astronomy, Stony Brook University
}

\altaffiltext{3}
{
  National Energy Research Scientific Computing Center, Lawrence Berkeley National Laboratory
}

\altaffiltext{4}
{
  Department of Physics and Astronomy, Michigan State University
}

\altaffiltext{5}
{
  Department of Civil and Environmental Engineering, University of Washington
}

\begin{abstract}
  This document contains the session description and extended abstract for
  a submission to GTC 2018.
\end{abstract}

\section{Session Description}

In this session you'll hear about one team's experience preparing an adaptive mesh
refinement library -- and a fluid dynamics code based on it -- for Summit, the
IBM POWER9 and NVIDIA Volta system at Oak Ridge National Laboratory, where multiple GPUs
are connected via NVLink to each other and the CPUs. It was simple to compile and run on
the OpenPOWER architecture, and to offload to the GPUs with CUDA Fortran, with little
architecture-specific code. Initial results with POWER8 and P100 have shown 
excellent CPU and GPU performance and good multi-node scaling for an astrophysics mini-app
that was difficult to run effectively on prior GPU architectures. We will also discuss
our experiences porting other modules in our multi-physics codes, and preliminary
results on the POWER9 and V100 platform.

\section{Extended Abstract}

% Put a citation in here just to get the references to compile

\cite{castro}

\section{Acknowledgements}

The work at Stony Brook was supported by DOE/Office of Nuclear
Physics grant DE-FG02-87ER40317 and NSF award AST-1211563.  An award
of computer time was provided by the Innovative and Novel
Computational Impact on Theory and Experiment (INCITE) program.  This
research used resources of the Oak Ridge Leadership Computing Facility
at the Oak Ridge National Laboratory, which is supported by the Office
of Science of the U.S. Department of Energy under Contract
No.\ DE-AC05-00OR22725.  We appreciate the efforts of the OLCF (and in
particular Fernanda Foertter) in organizing GPU hackathons.  This
research used resources of the National Energy Research Scientific
Computing Center, which is supported by the Office of Science of the
U.S. Department of Energy under Contract No.\ DE-AC02-05CH11231.
Development work benefited from a grant of a Titan X Pascal GPU
from NVIDIA through the GPU Grant Program, as well as access to
the CORAL Early Access systems: Summitdev at ORNL, and ray and rzmanta
at LLNL.

\bibliographystyle{aasjournal}
\bibliography{refs}

\end{document}
